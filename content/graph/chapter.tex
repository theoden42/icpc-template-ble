\chapter{Graph}

\section{Fundamentals}
	\kactlimport{BellmanFord.h}

\section{Network flow}
	\kactlimport{MinCostMaxFlow.h}

\section{Matching}
	\kactlimport{DFSMatching.h}
	\kactlimport{MinimumVertexCover.h}

\section{DFS algorithms}
	\kactlimport{SCC.h}
	\kactlimport{FindingBridges.h}
	\kactlimport{2sat.h}
	\kactlimport{EulerWalk.h}

\section{Coloring}
	\kactlimport{EdgeColoring.h}

\section{Heuristics}
	\kactlimport{MaximalCliques.h}
	\kactlimport{MaximumClique.h}

\section{Trees}
	\kactlimport{BinaryLifting.h}
	\kactlimport{LCA.h}
	\kactlimport{HLD.h}
	\kactlimport{DirectedMST.h}

\section{Math}
	\subsection{Number of Spanning Trees}
		% I.e. matrix-tree theorem.
		% Source: https://en.wikipedia.org/wiki/Kirchhoff%27s_theorem
		% Test: stress-tests/graph/matrix-tree.cpp
		Create an $N\times N$ matrix \texttt{mat}, and for each edge $a \rightarrow b \in G$, do
		\texttt{mat[a][b]--, mat[b][b]++} (and \texttt{mat[b][a]--, mat[a][a]++} if $G$ is undirected).
		Remove the $i$th row and column and take the determinant; this yields the number of directed spanning trees rooted at $i$
		(if $G$ is undirected, remove any row/column).

